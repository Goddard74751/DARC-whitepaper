\documentclass[main.tex]{subfiles}
\begin{document}



\section{Regulation and self-regulation in the crypto world}

Most of the DAOs deployed to blockchains such as Ethereum, Solana, and others are not supervised by any governments or laws. The smart contract will simply execute as specified by its programming. Rather than being supervised by formal processes and bodies, the DAOs are self-regulated by the entities deploying the smart contracts. Notably, they can decide to change the rules or update the smart contract of the DAOs by voting. Each DAO can function as a company, fund, or non-profit community. The token holders will vote and decide how to run the DAO. The DAO was invented by Christoph Jentzsch \cite{dao_paper}, the creator of The DAO concept and the author of the first DAO paper. Despite its initial divisiveness, the DAO concept is now very popular and widely used in the crypto world, leading to the Ethereum community split into the current "new" Ethereum and the original unforked "Ethereum Classic."

Though DAOs are self-regulated, they can still be regulated by governments, and many governments are trying to regulate the crypto world by providing legislation services for DAOs, regulating them under government and tax laws. This allows a DAO to run under the management of a registered company and engage in business activities such as providing services, making payments, and offering products. However, these regulated DAOs do not have the same level of freedom as fully self-regulated DAOs. Government-regulated DAOs are not built on top of a decentralized infrastructure, and governments can still regulate them through tax and stock market laws. While these regulated DAOs may not be as popular as self-regulated DAOs, governments are actively exploring various ways to regulate the crypto world.

Another way to run a DAO is without any legal entity, and most DAOs today operate in this fashion. All operations are proposed and voted on by the community token owners. The owners can buy tokens from decentralized exchanges (DEX) such as Uniswap and centralized exchanges (CEX) such as Coinbase, Binance, and FTX, and participate in the voting process to decide how to run the DAO. The core contributors and managers of these DAOs define the rules of the DAO in smart contracts, primarily using Solidity for Ethereum and Rust for Solana. They decide whether it's open source or private, define who can start a proposal, and set the voting rules such as proposal duration, voting approval threshold, and voting award with the minting of new tokens.

\subsection{Current designs of popular DAOs}

The design of DAOs has evolved over time, becoming more complex and incorporating additional features. As flaws, vulnerabilities, and inefficiencies were identified in older DAOs, current popular ones have emerged as resolutions to these issues. Some popular DAOs today include Aragon, Moloch, and DAOStack, among others.

One prominent feature among well-designed DAOs is the presence of a fully functional user interface for DAO governance. Efforts are made to reach out to individuals who are not familiar with blockchain coding or have limited coding knowledge, achieved through user-friendly interfaces. These DAOs require less technical expertise to participate without sacrificing any features for simplicity. Another common feature is the implementation of additional restrictions on a fundamental aspect of DAO governance: voting. For instance, Moloch adopts weighted voting instead of one vote per token, while Aragon utilizes a modular "court" as a dispute resolution mechanism to prevent 51\% attacks and malicious proposals. Most popular DAOs are also structured in a modular or compartmentalized manner, benefiting from enhanced organization and scalability, which are key focal points for many DAOs. Aragon, for example, incorporates various groups such as a group for all token holders, an executive sub-dao, a compliance sub-dao, and a tech group.

While these DAOs share several characteristics, they are differentiated by their unique key features. Moloch stands out in various aspects. Initially, it employed permissioned membership, allowing only existing members to vote on the addition of new members. Moloch also introduced a "ragequit" feature, enabling members to exchange all membership shares for treasury assets. Additionally, proposals required sponsorship before being eligible for voting, and a grace period was enforced for the execution of accepted proposals. Non-transferable power within the DAO is not necessarily a unique aspect. In Moloch v2, several new features were introduced, significantly shaping the current landscape of DAOs. These features include multi-class token support, allowing the use of up to 200 tokens instead of just one, funding proposals, open submissions for non-members, a "GuildKick" mechanism to remove malicious members, and more.

Another distinctive feature can be found in DAOStack, a popular DAO today. It incorporates ``holographic consensus,'' which addresses challenges arising from an overwhelming number of proposals and decision-making requirements. This approach enables a small group to vote on behalf of a larger majority, aligning with their desires. DAOStack also employs a separate metric called "reputation" instead of correlating tokens to voting power. This approach aims to prevent the emergence of a plutocracy, according to their belief.

Overall, DAOs are rapidly evolving, and their current designs vary significantly. There has been a notable shift in focus towards enhancing security and scalability. This shift stems from past and present instances of members losing their assets due to malicious actors exploiting non-restrictive and simplistic structures. Efforts are being made to enhance the security and robustness of voting, proposal submission, membership, and other aspects, alongside a strong emphasis on facilitating easy and efficient usage for all. However, as more DAOs emerge and investment in the DAO industry increases significantly, it becomes necessary for them to differentiate themselves from one another. As a result, we observe different features in Aragon, Moloch, and DAOStack, each attempting to address distinct challenges or create new tools. It is safe to say that there is still much room for improvement in DAOs, and we anticipate that future DAO designs will differ significantly from the present.


\subsection{``One token, one vote'' vs multi-class-token vote}

Most DAOs make decisions by creating proposals, submitting them to the community, and conducting community votes. Community members can vote by sending tokens to the DAO smart contract, which then counts the votes and determines the approval or rejection of the proposal. Each token represents one vote, and the majority of votes determines the outcome. However, the lack of regulations or restrictions on the ``majority'' of token owners in most DAOs can create significant safety and security risks. For example, if the co-founders hold 49.9\% of the total tokens and promise to sell the remaining 50.1\% to the community during project launch, they can regain control by purchasing 0.2\% of tokens anonymously. Without regulations beyond majority voting, the co-founders can mint new tokens or withdraw funds without limitations.

In comparison, modern corporations with stocks follow internal by-laws or public regulations regarding processes like ``follow-on public offerings (FPOs).'' Co-founders and major shareholders must disclose FPO information to the public, and FPOs are limited by the total number of shares. Government or regulatory agencies, such as the U.S. Securities and Exchange Commission, China Securities Regulatory Commission, or Hong Kong Securities and Futures Commission, regulate FPOs according to stock market laws. In private companies, co-founders and major shareholders can issue more shares through a process called "private placement," which is not regulated by stock market laws. However, they are still required to disclose information about the private placement to the public. Early-stage investors face strict restrictions on their shares, such as lock-up periods, vesting periods, and limits on the number of shares they can sell. Co-founders and major shareholders must also disclose information about these restrictions. Moreover, different shareholders and shares may have additional restrictions, such as non-dilution shares, anti-dilution shares, and super-voting shares, which must be disclosed to the public.

In modern stock corporations, while major issues require all-hands voting, daily operations are typically reviewed and approved by executives and the board of directors, rather than by a general vote from all shareholders. In contrast, many DAOs today solely rely on all-hands voting to approve community proposals, which can be inefficient and time-consuming. DAOs often set a voting threshold, such as 51\% of the votes, and a proposal is approved if the threshold is reached. They also establish a voting duration, such as one day, and a proposal is approved if the duration is met. Some DAOs offer a voting reward, such as 1\% of the total tokens, which is distributed to the majority voters. Additionally, to prevent excessive sequential proposals, DAOs usually set a minimum token ownership threshold for creating a proposal, for example, 10\%. Consequently, most token owners are not allowed to create proposals unless they meet the minimum token requirement. DAO proposal mechanisms can be designed as a blocking queue, where token owners must wait for the closure of previous proposals before creating new ones. This necessitates a multi-class hierarchy design for DAOs, enabling a multi-class-token voting system.

Another lesson from modern stock corporations is the concept of "super-voting shares." In joint-stock companies, super-voting shares have higher voting power than regular shares, providing owners with greater influence over decision-making processes within the company. Owners of super-voting shares may have more votes per share at shareholder meetings or the power to veto specific decisions. Companies often use super-voting shares to grant founders, executives, or key stakeholders more control over the company's direction. While this can align leadership with long-term interests, it may also create conflicts of interest and raise concerns about accountability and fairness.

The use of super-voting shares in corporate governance can be controversial, as it gives a small group of individuals disproportionate influence over the company. Consequently, laws and policies regulate the use of super-voting shares to balance the interests of different stakeholders. It is worth considering this concept as a potential way to design DAOs, although the use of super-voting shares should be approached carefully, taking into account corporate governance principles.

\subsection{By-laws and procedure}

The by-laws of a company consist of the rules and regulations that govern its internal operations and management. These by-laws cover various topics, including the organizational structure, officer and director responsibilities, meeting and voting procedures, and rules for share issuance and transfer.

Typically, the board of directors adopts the company's by-laws, which must then be approved by the shareholders. The purpose of these by-laws is to establish a clear and consistent framework for the company's governance and decision-making processes. They are regularly reviewed and updated to align with any operational or legal changes.

In some cases, external regulations from government agencies or industry bodies may also apply to a company's by-laws. For instance, publicly traded companies must ensure compliance with securities laws and other relevant regulations. Nonetheless, by-laws primarily serve as an internal document to ensure the company's smooth and effective operation.

In a self-regulated corporation, the by-laws focus on defining what actions are prohibited rather than what is permissible. They provide procedures for resolving potential issues that may arise in the future.

However, in the current design of DAOs, there is typically only one procedure: majority voting. Most DAOs only impose restrictions or regulations related to the voting process, such as voting thresholds, durations, awards, and minimum token requirements for proposal creation. However, there is no by-law that outlines prohibited actions or establishes procedures for addressing potential issues. As a result, major shareholders can easily control DAOs by proposing actions that harm minority shareholders, such as diluting their token holdings, or by executing "rug pull" proposals to drain funds from the DAOs. In this context, DAOs lack self-regulation through smart contract codes or community oversight. With decision-making solely dependent on majority voting, major shareholders can manipulate DAOs and create proposals that benefit themselves at the expense of others, leading to a "Tyrannical Majority" scenario.

In healthy, stable, and robust business entities like joint-stock companies or non-profit organizations, by-laws are crucial for defining what actions are prohibited and establishing procedures to address potential future problems. Shareholders, core executives, and the board of directors must set the boundaries of operations and rights within the organization before joining or investing. This enables them to prevent conflicts of interest and address potential problems. Similarly, in the design of DAOs, by-laws are essential to impose additional restrictions on different token owners before other token owners can purchase tokens from them. This approach goes beyond simple token voting, as relying solely on voting for all proposals cannot solve all problems and may even create further issues through the passage of "Tyrannical Majority" proposals.

\subsection{``Tokenomics'' vs ``Dividend''}

For DAOs and other crypto projects (SocialFi, GameFi, or other protocols), tokenomics is often used as a means to attract and retain investors, as compared to dividends. Tokenomics refers to the study of designing and implementing tokens within a cryptocurrency system. It involves creating a token, which is a digital asset used for exchange within a specific cryptocurrency ecosystem. Tokens represent value within the system, enabling the purchase and sale of goods and services, as well as incentivizing desired behaviors. The design of a token's economy is crucial for the overall success of a cryptocurrency, as it determines its utility and value over time. A well-designed token economy helps ensure the long-term viability and stability of the cryptocurrency.

This leads us to the question: are tokens in DAOs and crypto projects the same as stocks in traditional companies? Is purchasing or earning tokens equivalent to investing in stocks? Are token holders considered investors or customers? Are tokens comparable to stocks for investment purposes or products for sale?

In real-world companies regulated by the U.S. SEC, no company exists solely to sell ``stock'' to the market, and issuing 10 to 1000 times the stock without regulation is not feasible. There is a clear distinction between ``stock'' and ``product'': ``stock'' represents ownership in a company, encompassing not only ownership of assets and market value but also the right to participate in the company's decision-making process through specific procedures outlined in the company's by-laws, contracts, and regulations. Conversely, ``product'' refers to the goods or services produced and sold. It can encompass specific items offered for sale or the overall offering made to customers. In essence, companies generate revenue by creating goods or services as "products" for sale, while raising funds by issuing and selling new stock in primary or secondary markets, thereby diluting the ownership of existing shareholders. This mechanism forms the core of the traditional stock market, driving shareholders, executives, employees, and board members to collaborate and create tangible value while deriving profits from genuine business activities.

In most tokenomic designs, a precise definition of tokens is often lacking. Many DAOs and crypto projects utilize tokens to attract and retain investors, and these tokens can be traded on centralized (CEX) or decentralized (DEX) exchanges with voting power. However, tokens are not equivalent to stocks for investment purposes in most tokenomic designs. Firstly, in the majority of tokenomic algorithms, the core smart contract continually mints new tokens as rewards to SocialFi users, GameFi users, and DAO voters. This substantial increase in token supply leads to value dilution, reducing the significance and ownership weight of individual tokens. Secondly, smart contracts generally lack regulations or procedures guaranteeing the rights of token owners, apart from the "one-token one-vote" principle, which is often inefficient and time-consuming. It fails to prevent co-founders from executing rug pulls, in which they fraudulently withdraw assets belonging to other shareholders. Lastly, as observed in various cases, if co-founders can raise sufficient funds by minting and selling tokens through a ``tokenomics'' approach, they are likely to halt actual business expansion, allowing them to abscond with the funds (also known as a "rug pull") and abandon the project. Consequently, tokens are not equivalent to stocks for investment purposes, and the project's "business" revolves solely around "selling tokens" rather than creating and selling tangible goods or services. Consequently, token owners who purchase tokens for investment purposes differ from stock investors, despite believing in the illusion of a false ``tokenomics'' approach and anticipating token price appreciation with project growth, which typically yields opposite outcomes.

In a healthy and well-regulated stock market, most companies distribute dividends to their shareholders. Over 80\% of companies listed in the S\&P 500 index pay dividends periodically. Dividends hold significance for traditional stock companies as they allow companies to share their profits with shareholders. When a company generates a profit, it has various options for utilizing that money. It can reinvest the funds in the business to facilitate growth and expansion, pay off debts, or distribute a portion of the profits to shareholders in the form of dividends. Paying dividends helps companies attract and retain investors seeking regular income from their investments. Additionally, consistent dividend payments indicate financial stability and strength, enhancing investor confidence in the company. Some companies, particularly those in high-growth industries or prioritizing reinvestment of profits for fueling growth, may choose not to pay dividends.

Several reasons exist for companies opting not to pay dividends. Firstly, a company may lack surplus cash to distribute to shareholders. If a company is heavily investing in its operations, research and development, or other growth opportunities, it may not have available cash for dividend payments. In such cases, the company may retain profits to reinvest in the business and drive future growth.

Secondly, a company may believe that paying dividends is not the most effective way to generate value for its shareholders. For instance, a fast-growing company might believe that reinvesting profits in the business creates more value for shareholders than distributing dividends. In such cases, the company focuses on revenue and earnings growth, expecting this to result in higher share prices and better returns for shareholders.

Furthermore, some companies choose to forgo dividends to retain flexibility in capital allocation decisions. By retaining profits instead of paying dividends, companies can utilize their cash for acquisitions, strategic investments, or seizing other opportunities as they arise. This grants them greater control over their future and aids in building a stronger and more resilient business.

In conclusion, ``tokenomics'' is not the sole or appropriate approach for designing crypto projects or DAOs. Dividends offer a healthier and more stable method to stabilize token/stock prices in the market, as opposed to "infinitely minting tokens." However, dividends are not widely used in current crypto projects.



\end{document}