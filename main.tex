\documentclass{article}

% Language setting
% Replace `english' with e.g. `spanish' to change the document language
\usepackage[english]{babel}

% Set page size and margins
% Replace `letterpaper' with `a4paper' for UK/EU standard size
\usepackage[letterpaper,top=2cm,bottom=2cm,left=3cm,right=3cm,marginparwidth=1.75cm]{geometry}

% Useful packages
\usepackage{amsmath}
\usepackage{graphicx}
\usepackage{algorithm}
\usepackage{algorithmic}
\usepackage[colorlinks=true, allcolors=blue]{hyperref}
\usepackage{tikz-qtree}
\usepackage{tikz}
\usetikzlibrary{arrows.meta,bending,positioning}
\usepackage{listings}
\usepackage{subfiles}
\usepackage{appendix}
\usepackage{tabularx}
\usepackage{multirow}
\usepackage{spverbatim}
\usepackage{makecell}
\usepackage{authblk}
\usepackage{blindtext}



\title{DARC: Decentralized Autonomous Regulated Corporation}


\author[1]{Xinran Wang}
\author[1]{Guyang Li}
\author[2]{Tong Che}
\author[3]{Yiran Su}




\affil[1]{School of Computing, University of Utah, Salt Lake City, UT, USA \authorcr
  \{\tt peter, cleveland, joe\}@utah.edu}
\affil[2]{NVIDIA Research, Santa Clara, CA, USA \authorcr
  \{\tt glenn, joe\}@nvidia.com}
\affil[3]{Department of Engineering, Harvey Mudd College, Claremont, CA, USA \authorcr
  \tt joe@hmc.edu}


\begin{document}

\maketitle

\footnotetext[0]{In any scholarly discourse concerning the Decentralized Autonomous Regulated Corporation (DARC), encompassing academic papers, journals, books, documents, conference papers, manuals, reports, analyses, slides, as well as any literature or outputs of comparable nature, and further including any literature or outputs emerging from research and projects of academic, commercial, educational, governmental, or other affiliations either directly or indirectly sponsored or invested in the form of DARC, and any literature or outputs of analogous nature arising directly or indirectly from the utilization of any products associated with Project DARC, this paper imposes a mandatory requirement for authors or the institutional entities, groups, organizations, governments, schools, and the like involved in the projects to conscientiously incorporate this paper into the list of cited references in the scholarly literature.}

\footnotetext[1]{The architecture of DARC, interface, opcodes, condition nodes, plugins, by-law script, and other designs introduced in this paper are provided as reference designs. The final released version should be referenced from the implementation in the DARC code repository on GitHub: \url{https://github.com/project-darc/darc}}

\begin{abstract}
Decentralized autonomous organizations (DAOs) show promise but lack governance mechanisms. This paper proposes DARC, a Decentralized Autonomous Regulated Corporation designed to facilitate oversight of DAOs. DARC operates as a virtual machine on an EVM-compatible blockchain and incorporates configurable ``plugins'' outlining laws and rules. Users can run programs on the DARC virtual machine to execute company operations such as equity, cash, and voting, with all actions subject to constraints and rules defined by plugins. DARC supports multi-token systems and dividends distribution, resembling corporate features. The By-law Script serves as DARC's programming language, enabling the design of DARC programs and operations, as well as the customization of plugins.
\end{abstract}

\section{Introduction}


Ethereum \cite{buterin2013ethereum}, launched in 2015, is a decentralized blockchain platform known for introducing smart contracts. These self-executing contracts enable trustless and automated transactions. Ethereum's contributions include fostering decentralized applications (dApps) and innovations like decentralized finance (DeFi) and non-fungible tokens (NFTs). Decentralized autonomous organizations (DAOs)\cite{jentzsch2016decentralized} have emerged as a novel form of internet-native organization mediated by tokenized governance systems. By leveraging cryptographic primitives from public blockchains, participation rights can be embedded directly into cryptoassets/tokens and allocated algorithmically without traditional corporate structures. However, early experiments like The DAO have exposed risks around security, flexibility, and real-world applicability.

This paper introduces DARC - the Decentralized Autonomous Regulated Corporation. DARC incorporates modular ``plugins'' that encode rules and policies analogous to corporate bylaws. A multi-token system allows flexible rights allocation, mimicking shares. And a virtual machine architecture enables oversight and control of operations.

By synthesizing aspects from corporate structures and DAO architectures, DARC offers a regulated and adaptable foundation for decentralized organizations seeking real-world coordination. It bridges the gap between traditional corporations and autonomous DAOs.

The paper begins by delineating the key principles and design rationale in developing DARC. The overall system architecture is explained followed by specifics on essential sub-components: plugins, multi-token model, sandboxed execution, voting apparatus and others. Salient examples are also provided to illustrate the flexible configurability afforded by DARC across use cases spanning corporate stocks, bonds, boards, upgrades and emergency response among others.

The paper concludes by reflecting on future directions for Decentralized Autonomous Regulated Corporations in catalyzing a new paradigm of organizational structures and economic coordination on the blockchain.


\subfile{sub_principles}


\subfile{sub_architecture}


\subfile{sub_bylawscript}




\subfile{sub_multi_token_system}


\subfile{sub_plugins}





\subfile{sub_voting}

\subfile{sub_memberships}

\subfile{sub_dividends}

\subfile{sub_emergency}

\subfile{sub_upgradablity}

\subfile{sub_discussion}


\bibliographystyle{alpha}
\bibliography{sample}

\subfile{sub_appendix}

\end{document}