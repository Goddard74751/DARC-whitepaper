\documentclass{article}

% Language setting
% Replace `english' with e.g. `spanish' to change the document language
\usepackage[english]{babel}

% Set page size and margins
% Replace `letterpaper' with `a4paper' for UK/EU standard size
\usepackage[letterpaper,top=2cm,bottom=2cm,left=3cm,right=3cm,marginparwidth=1.75cm]{geometry}

% Useful packages
\usepackage{amsmath}
\usepackage{graphicx}
\usepackage{algorithm}
\usepackage{algorithmic}
\usepackage[colorlinks=true, allcolors=blue]{hyperref}
\usepackage{tikz-qtree}
\usepackage{tikz}
\usetikzlibrary{arrows.meta,bending,positioning}
\usepackage{listings}
\usepackage{subfiles}
\usepackage{appendix}
\usepackage{tabularx}

\usepackage{spverbatim}

\title{DARC: Decentralized Autonomous Regulated Corporation}
\author{Xinran Wang}

\begin{document}
\maketitle

\begin{abstract}
Decentralized autonomous organizations (DAOs) show promise but lack governance mechanisms. This paper proposes DARC, a Decentralized Autonomous Regulated Corporation designed to facilitate oversight of DAOs. DARC operates as a virtual machine on an EVM-compatible blockchain and incorporates configurable ``plugins'' outlining laws and rules. Users can run programs on the DARC virtual machine to execute company operations such as equity, cash, and voting, with all actions subject to constraints and rules defined by plugins. DARC supports multi-token systems and dividends distribution, resembling corporate features. The By-law Script serves as DARC's programming language, enabling the design of DARC programs and operations, as well as the customization of plugins.
\end{abstract}

\section{Introduction}


Decentralized autonomous organizations (DAOs) offer promise but lack governance mechanisms. This paper proposes DARC, a Decentralized Autonomous Regulated Corporation enabling oversight of DAOs. DARC incorporates configurable "plugins" outlining policies and procedures. 

Your introduction goes here! Simply start writing your document and use the Recompile button to view the updated PDF preview. Examples of commonly used commands and features are listed below, to help you get started.

Once you're familiar with the editor, you can find various project settings in the Overleaf menu, accessed via the button in the very top left of the editor. To view tutorials, user guides, and further documentation, please visit our \href{https://www.overleaf.com/learn}{help library}, or head to our plans page to \href{https://www.overleaf.com/user/subscription/plans}{choose your plan}.


\subfile{sub_principles}


\subfile{sub_architecture}


\subfile{sub_bylawscript}





\subfile{sub_multi_token_system}

\subfile{sub_plugins}





\subfile{sub_voting}

\subfile{sub_memberships}

\subfile{sub_dividends}

\subfile{sub_emergency}

\subfile{sub_upgradablity}

\section{Conclusion}


\bibliographystyle{alpha}
\bibliography{sample}

\subfile{sub_appendix}

\end{document}